\chapter*{Введение}
\addcontentsline{toc}{chapter}{Введение}

В данной работе будет рассмотрена проблема классификации и генерации сигналов дистанционно-управляемых аппаратов.

Способность классифицировать и генерировать сигналы дистанционно-управляемых аппаратов относится к важнейшим функциональным возможностям систем подавления и захвата беспилотных аппаратов в настоящее время.

\textbf{Цель данной научно-исследовательской работы} --- провести обзор методов классификации и генерации шумов и помех с помощью нейронных сетей.


Для достижения поставленной цели необходимо решить следующие задачи:
\begin{itemize}
	\item[-] классифицировать основные типы сигналов и модуляций;
	\item[-] изучить основные методы формирования помехи;
	\item[-] определить базовое описание и разделение машинного обучения и глубокого обучения;
	\item[-] представить различные типы нейронных сетей;
	\item[-] сравнить представленные нейронные сети в рамках классификации генерации шума для формирования помехи.
\end{itemize}